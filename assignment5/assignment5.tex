\documentclass{article}
\usepackage{graphicx}
\usepackage{amsmath}

\author{Gui Wenxuan, 2016302580142}
\title{Assignment 5}
\begin{document}
   \maketitle
  
\newpage
\section{Problem1 and 2} Looking at Figure 5.3, enumerate the paths from y to u that do not contain any loops\\
And repeat Problem P1 for paths from x to z, z to u, and z to w.\\

\begin{figure}[htbp]
\begin{center}
\includegraphics[scale=0.5]{Figure5_3.png}
\caption{Figure 5.3}
\label{Figure 5.3}
\end{center}
\end{figure}



\noindent Solution:\\
\begin{enumerate}
\item \emph{y to u} \\
\begin{figure}[htbp]
\begin{center}
\includegraphics[scale=0.1]{y2u.jpg}
\caption{y to u}
\label{default}
\end{center}
\end{figure}
\item \emph{x to z} \\
\begin{figure}[htbp]
\begin{center}
\includegraphics[scale=0.1]{x2z.jpg}
\caption{x to z}
\label{default}
\end{center}
\end{figure}
\item \emph{z to u} \\
\begin{figure}[htbp]
\begin{center}
\includegraphics[scale=0.1]{z2u.jpg}
\caption{z to u}
\label{default}
\end{center}
\end{figure}
\item \emph{z to w} \\
\begin{figure}[htbp]
\begin{center}
\includegraphics[scale=0.1]{z2w.jpg}
\caption{z to w}
\label{default}
\end{center}
\end{figure}
\end{enumerate}

\newpage
\section{Problem 3} Consider the following network. With the indicated link costs, use Dijkstra’s shortest-path algorithm to compute the shortest path from x to all network nodes. Show how the algorithm works by computing a table similar to Table 5.1.\\
\begin{figure}[htbp]
\begin{center}
\includegraphics[scale=0.5]{Problem3.png}
\caption{Problem3}
\label{default}
\end{center}
\end{figure}

\noindent Solution:
\begin{center}
  \begin{tabular}{@{} cccccccc @{}}
    \hline
    step & N' & D(v),p(v) & D(w),p(w) & D(y),p(y) & D(z),p(z) & D(t),p(t) & D(u),p(u) \\ 
    \hline
    0 & x & 3,x & 6,x & 6,x & 8,x & max & max \\ 
    1 & xv &  & 6,x & 6,x & 8,x & 7,v & 6,v \\ 
    2 & xvu &  & 6,x & 6,x & 8,x & 7,v &  \\ 
    3 & xvuw &  &  & 6,x & 8,x & 7,v &  \\ 
    4 & xvuwy &  &  &  & 8,x & 7,v &  \\ 
    5 & xvuwyt &  &  &  & 8,x &  &  \\ 
    6 & xvuwytz &  &  &  &  &  &  \\ 
    \hline
  \end{tabular}
\end{center}

\newpage
\section{Problem 5} Consider the network shown below, and assume that each node initially knows the costs to each of its neighbors. Consider the distance-vector algo- rithm and show the distance table entries at node z.\\
\begin{figure}[htbp]
\begin{center}
\includegraphics[scale=0.5]{Problem5.png}
\caption{Problem5}
\label{default}
\end{center}
\end{figure}

\noindent Solution:\\
\begin{table}[htbp]
  \centering
  \begin{tabular}{@{} ccc @{}}
    \hline
    destination & hit & cost \\ 
    \hline
    x &   & 2 \\ 
    y & x & 5 \\ 
    u & x-v & 6 \\ 
    v & x & 5 \\ 
    \hline
  \end{tabular}
  \caption{TableCaption}
  \label{tab:label}
\end{table}


\newpage
\section{Problem 12} What is the message complexity of LS routing algorithm? \\

\noindent Solution: We have seen that LS requires each node to know the cost of each link in the network. This requires O(\big|N\big| \big|E\big|) messages to be sent. Also, whenever a link cost changes, the new link cost must be sent to all nodes. With n nodes, with an average of l links/node, each node sends O($n$l). Total messages O($n^2$l)\\





\newpage
\section{Problem 14} Consider the network shown below. Suppose AS3 and AS2 are running OSPF for their intra-AS routing protocol. Suppose AS1 and AS4 are running RIP for their intra-AS routing protocol. Suppose eBGP and iBGP are used for the inter-AS routing protocol. Initially suppose there is no physical link between AS2 and AS4.\\
a. Router 3c learns about prefix x from which routing protocol: OSPF, RIP, eBGP, or iBGP?\\
b. Router 3a learns about x from which routing protocol?\\
c. Router 1c learns about x from which routing protocol?\\
d. Router 1d learns about x from which routing protocol?\\
\begin{figure}[htbp]
\begin{center}
\includegraphics[scale=0.5]{Problem14.png}
\caption{Problem14}
\label{default}
\end{center}
\end{figure}
 

\noindent Solution:\\
\begin{enumerate}
\item eBGP because 3c spans two ASs
\item iBGP because 3a learns about x from 3c, they are in the same AS.
\item eBGP
\item iBGP because 1a and 1b are in the same AS and 1d is between them
\end{enumerate}




\end{document}