\documentclass{article}
\usepackage{graphicx}
\usepackage{amsmath}

\author{Gui Wenxuan, 2016302580142}
\title{Assignment 1}
\begin{document}
   \maketitle
   ~\\
   Q1: ping another computer\\
   A1: \begin{figure}[ht]
	\centering
	\includegraphics[scale=0.25]{ping.jpg}
	\caption{default}
	\label{ping}
	\end{figure}\\
  Q2: traceroute a server\\
  A2: \begin{figure}[ht]
	\centering
	\includegraphics[scale=0.25]{traceroute.png}
	\caption{default}
	\label{traceroute}
	\end{figure}
\newpage
\section{Problem7} In this problem, we consider sending real-time voice from Host A to Host B over a packet-switched network (VoIP). Host A converts analog voice to a digital 64kbps bit stream on the fly. Host A then groups the bits into 56-byte packets. There is one link between Hosts A and B; its transmission rate is 2Mbps and its propagation delay is 10ms. As soon as Host A gathers a packet, it sends it to Host B. As soon as Host B receives an entire packet, it converts the packet’s bits to an analog signal. How much time elapses from the time a bit is created (from the original analog signal at Host A) until the bit is decoded (as part of the analog signal at Host B)?\\


  Solution: time elapses = $56*8 bits / 64*10^3 + 56*8 / 2*10^3 + 10 = 17.224$ ms\\
\newpage
\section{Problem8} Suppose users share a 3 Mbps link. Also suppose each user requires 150 kbps when transmitting, but each user transmits only 10 percent of the time.\\
  \begin{enumerate}
  \item When circuit switching is used, how many users can be supported?
  \item For the remainder of this problem, suppose packet switching is used. Find the probability that a given user is transmitting.
  \item suppose there are 120 users. Find the probability that at any given time, exactly $n$ users are transmitting simultaneously.
  \item Find the probability that there are 21 or more users transmitting simultaneously.
  \end{enumerate}
  
  
  Solution: 
  \begin{enumerate}
  \item 3 Mbps / 150 kbps = 20 users can be supported.
  \item $p$ = 0.1
  \item $p = \binom{120}{n}\ p^n(1-p)^{120-n}$
  \item $p = 1 - \sum_{n=0}^{20} \binom{120}{n}\ p^n(1-p)^{120-n}$ 
  \end{enumerate}
\newpage
\section{Problem9} Consider the discussion in Section 1.3 of packet switching versus circuit switching in which an example is provided with a 1 Mbps link. Users are generating data at a rate of 100 kbps when busy, but are busy generating data only with probability p = 0.1. Suppose that the 1 Mbps link is replaced by a 1 Gbps link.
  \begin{enumerate}
  \item What is N, the maximum number of users that can be supported simultaneously under circuit switching?
  \item Now consider packet switching and a user population of M users. Give a formula (in terms of p, M, N) for the probability that more than N users are sending data.
  \end{enumerate}
  
  
  Solution: 
  \begin{enumerate}
  \item $N = 1Gbps / 100kbps / p = 10000$
  \item $p = \sum_{n=N+1}^{M} \binom{M}{n}\ p^n(1-p)^{M-n}$
  \end{enumerate}
\end{document}